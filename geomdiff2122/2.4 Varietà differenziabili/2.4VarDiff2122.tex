%!TEX encoding = UTF-8 Unicode

\documentclass[80pt,defrtgr]{report}
\usepackage[utf8]{inputenc}
\usepackage[T1]{fontenc}
\usepackage[portuguese]{babel}
\usepackage{amsmath}
\usepackage{amssymb}
\usepackage{fullpage}
\usepackage{color}


\def\Der{\mathop{\hbox{Der}}}

\def\u{\mathop{\hbox{vol}}}
\def\v{\mathop{\hbox{vol}}}
\def\w{\mathop{\hbox{vol}}}
\def\0{\mathop{\hbox{0}}}
\def\normalshape{\rm}

\def\sin{\mathop{\hbox{sin}}}
\def\cos{\mathop{\hbox{cos}}}
\def\tg{\mathop{\hbox{tg}}}
\def\vol{\mathop{\hbox{vol}}}
\def\det{\mathop{\hbox{det}}}
\def\tr{\mathop{\hbox{tr}}}
\def\curv{\mathop{\hbox{curv}}}
\def\hol{\mathop{\hbox{hol}}}
\def\Aut{\mathop{\hb|x{Aut}}}
\def\Isom{\mathop{\hbox{Isom}}}
\def\Hom{\mathop{\hbox{Hom}}}
\def\dim{\mathop{\hbox{dim}}}
\def\GL{\mathop{\hbox{GL}}}
\def\PGL{\mathop{\hbox{PGL}}}
\def\PU{\mathop{\hbox{PU}}}
\def\Pic {\mathop{\hbox{Pic}}}
\def\O {\mathop{\hbox{O}}}
\def\SO {\mathop{\hbox{SO}}}
\def\SL {\mathop{\hbox{SL}}}
\def\U {\mathop{\hbox{U}}}
\def\u {\mathop{\hbox{u}}}
\def\H {\mathop{\hbox{H}}}
\def\S {\mathop{\hbox{S}}}
\def\M {\mathop{\hbox{M}}}
\def\tr {\mathop{\hbox{tr}}}
\def\Symp {\mathop{\hbox{Symp}}}
\def\dimn {\mathop{\hbox{dim}}}
\def\volm {\mathop{\hbox{vol}}}
\def\span {\mathop{\hbox{Span}}}
\def\End {\mathop{\hbox{End}}}
%\def\R {\mathop{\hbox{Re}}}
\def\I {\mathop{\hbox{Im}}}
\def\Arg {\mathop{\hbox{Arg}}}

\newtheorem{teor}{Teorema}
\newtheorem{esempio}[teor]{Esempio}
\newtheorem{osservazione}[teor]{Osservazione}
\newtheorem{defin}[teor]{Definizione}
\newtheorem{nota}[teor]{Nota}
\newtheorem{prop}[teor]{Proposizione}
\newtheorem{corol}[teor]{Corollario}
\newtheorem{lemma}[teor]{Lemma}
\newtheorem{guess}[teor]{Congettura}
\newtheorem{proprieta}[teor]{Fatto}
\newtheorem{problem}[teor]{Problema}

\newcommand{\fdim}{\hspace*{\fill}$\Box$}
\newcommand{\dimostr}{{\bf Dimostrazione: }}

\newcommand{\K}{\mathbb{K}}
\newcommand{\R}{\mathbb{R}}
\newcommand{\complex}{\mathbb{C}}
\newcommand{\intero}{\mathbb{Z}}
\newcommand{\projective}{\mathbb{P}}
\newcommand{\hyperbolic}{\mathbb{H}}
\newcommand{\natur}{\mathbb{N}}
\newcommand{\razionale}{\mathbb{Q}}
\newcommand{\quaternione}{\mathbb{H}}
\newcommand{\campo}{\mathbb{K}}


\renewcommand{\baselinestretch}{1.2}
\footskip65pt

% \`e  \`a
\begin{document}
\centerline{\bf Varietà Differenziabili (seconda parte)}

\centerline{\bf Corso di Laurea in Matematica A.A. 2021-2022}

\centerline {\bf Docente:
Andrea Loi}



\vspace{0.5cm}
\begin{enumerate}
\item
Siano $S_1$ e $S_2$ due sottovariet\`a di due variet\`a differenziabili $M_1$ e $M_2$ rispettivamente. Dimostrare che $S_1\times S_2$
\'e una sottovariet\`a di $M_1\times M_2$. Dedurre che il toro $T^n$ il toro di dimensione $n$ è una sottovarietà di $\R^{2n}$.
\item
Sia $F:\R^2\rightarrow \R, (x, y)\mapsto x^2-6xy+y^2$. Trovare i $c\in\R$ tali che $F^{-1}(c)$ sia una sottovariet\`a di $\R^2$.
\item
Dire se le soluzioni del sistema 
$$\left\{ \begin{array}{l}
 x^3+y^3+z^3=1\\
z=xy
\end{array}\right.$$
costuiscono una sottovariet\`a di $\R^3$. 
\item
Un  polinomio $F(x_0, \dots , x_n)\in \R[x_0, \dots , x_n]$ \'e omogeneo di grado $k$ se \'e combinazione lineare
di monomi $x_0^{j_1}\dots x_0^{j_m}$ di grado $k$, $\sum_{j=1}^mi_j=k$.
Dimostrare che
$$\sum_{i=0}^nx^i\frac{\partial F}{\partial x_i}=kF.$$
Dedurre che $F^{-1}(c)$, $c\neq  0$ \'e una sottovariet\`a di $\R^n$
di dimensione $n-1$.
Dimostrare, inoltre che per $c, d>0$, $F^{-1}(c)$ e $F^{-1}(d)$ sono diffeomorfe e lo stesso vale per $c, d<0$.
(Suggerimento per la prima parte: usare l'uguaglianza $F(\lambda x_0, \dots , \lambda x_n)=\lambda^kF(x_0, \dots , x_n)$ valida per ogni $\lambda\in\R$).
\item
Dimostare che
 $SL_n(\complex)=\{A\in GL_n(\complex) \ | \ \det A=1\}$
 \'e una sottovariet\`a di $M_n(\complex)$ di dimensione $2n^2-2$. 
 \item
 Sia $F:N\rightarrow M$ un'applicazione liscia tra varietà differenziabili. Dimostare che l'insieme  $PR_F$ dei punti regolari di $F$ \'e un aperto di $N$.
 \item
 Sia $F:N\rightarrow M$ un'applicazione liscia tra varietà differenziabili. Dimostrare che se $F$ \'e chiusa allora $VR_F$ (insieme dei punti regolari di $F$) \'e aperto in $M$.
 \item
 Dimostrare che  $F:\R\rightarrow \R^3, t\mapsto (t, t^2, t^3)$ \'e un embedding liscio  e scrivere $F(\R)$ come zero di funzioni.
 \item
 Dimostrare che  $F:\R\rightarrow \R^2, t\mapsto (\cosh t, \sinh t)$ \'e un embedding liscio  e $F(\R)=\{(x, y)\in\R^2 \ | \ x^2-y^2=1\}$.
 \item
 Dimostare che la composizione di immersioni \'e un'immersione e che il prodotto cartesiano di due immersioni \'e un'immersione.

 \item
 Dimostrare che se $F:N\rightarrow M$ \'e un'immersione e $Z\subset N$ \'e una sottovariet\`a di $N$ allora 
 $F_{|Z}:Z\rightarrow M$ \'e un'immersione.
 \item
Dimostrare che l'applicazione
 $$F:S^2\rightarrow \R^4, (x, y, z)\mapsto (x^2-y^2, xy, xz, yz)$$
 induce un embedding liscio da $\R P^2$ a $\R^4$.
 \item
 Dimostrare che un'immersione iniettiva e propria \'e un embedding liscio. Mostrare che esistono embedding lisci che non sono applicazioni  proprie.
 (Ricorda che un'applicazione continua $f:X\rightarrow Y$ tra spazi topologici è propria se $f^{-1}(K)$ è compatto in $X$ per ogni compatto $K$ di $Y$).
 %\item
 %Sia $M$ una sottovarietà di $\R^N$ di dimensione $n$. Dimostrare che per ogni $p\in M$
 %esiste un intorno  $V$ di $p$  in $\R^N$ e $g_{n+1}, \dots , g_{N}$ funzioni differenziabili da un aperto $U\subset \R^n$ in $\R$
 %tali che 
 %$$V\cap M=\{\left(a_1, \dots a_n, g_{n+1}(a), \dots ,g_N(a)\right) \ | \ a:=(a_1, \dots a_n)\in U \}.$$
 %In altre parole ogni sottovarietà di $\R^N$ è localmente esprimibile come grafico.
% (Suggerimento: usare il fatto (dimostrato a lezione) che se $M$ è una sottovarietà di $\R^N$ allora per ogni punto $p\in M$ si possono scegliere 
 %$n$   funzioni coordinate  $r^{i_1}, \dots ,r^{i_n}$ di $\R^N$ per definire un sistema di coordinate di $M$ in un intorno di $p$).
 
 \end{enumerate}
\end{document}
















\end{document}
